\newcommand{\doctitle}{行列のパーマネントを求める}
\newcommand{\dockeyword}{matrix permanent}
\newcommand{\docauthor}{Tamaki ISII}
\documentclass{mystyle}
\usepackage[fleqn]{mathpack}
\usepackage{mathpacksetup}
\usepackage{mathpackops}

\def\flst{\rlap{\quad。}}
\def\xflst{\hbox{\quad。}}

\newcounter{eq}[section]
\def\mathpackstdtag{\theeq}
\def\mathpackbeforestdtag{\global\stepcounter{eq}}
\def\mathpackstdtagtemplate#1{\rm (\thesection.#1)}
\def\mathpackstdlabeltemplate#1{eq.~\thesection.#1}

\leftskip=0pt

\usepackage[dvipdfmx]{graphicx}

\begin{document}
%
\mymaketitle
%
\begin{section}{定義及び問題提起}
\thmbegin{def}{\label{deffm}}
\linesenv{\prenobreak\postnobreak}{\notag}{%
  \showauto{\hbox{関数}\func{fm}\colon\setN\rightarrow\setN\hbox{を次のように定める。}}%
  \showauto{\func{fm}(j)\defeq\left\lbrace%
    \mathalign{\dispcenter\sep\mathleft}{%
      \showline{\frac{3j}{2}    & \textif\equivmod{j}{0}{4}}%
      \showline{\frac{3j+1}{4}  & \textif\equivmod{j}{1}{8}}%
      \showline{\frac{j+2}{4}   & \textif\equivmod{j}{2}{8}}%
      \showline{\frac{3j-1}{2}  & \textif\equivmod{j}{3}{4}}%
      \showline{\frac{3j+1}{8}  & \textif\equivmod{j}{5}{16}}%
      \showline{\frac{3j+2}{4}  & \textif\equivmod{j}{6}{8}}%
      \showline{\frac{3j+9}{16} & \textif\equivmod{j}{13}{32}}%
      \showline{\frac{j+3}{16}  & \textif\equivmod{j}{29}{32}}%
    }\right.%
  }%
}
\thmend
%
\thmbegin{def}{\label{defM}}
\linesenv{\prenobreak\postnobreak}{\notag}{%
  \showauto{M_k\hbox{を}k\times k\hbox{行列として下記のように定める。}}%
  \showauto{M_k[i,j]\defeq\left\lbrace
    \mathalign{\dispcenter\sep\mathleft}{%
      \showline{1 & \textif i=\func{fm}(j)\ror i=j}%
      \showline{0 & \textotherwise}%
    }\right.
  }%
}
\thmend
%
\thmbegin{conj}{\label{conj1}}
\singledispenv{\prenobreak\postnobreak}{\notag}{\rmfunc{perm}(M_k)=1}
\thmend
%
ここで, \mathref{conj1}を証明したい。
\end{section}
%
\begin{section}{ルーク問題}
%
\thmbegin{def}{\label{defv}}
非負の整数$n$と$0$,$1$のみを成分に持つ行列$A$に対して, $v_n(A)$を$A$の$1$の部分に互いにとられないよう$n$個の飛車を置く方法の総数と定義する。ただし, $n=0$の場合は$1$とする。
\thmend
%
\thmbegin{def}{\label{defr}}
$0$,$1$のみを成分に持つ$n\times n$行列$A$に対して, 多項式$R_A(x)$を以下のように定める。
\singledispenv{\prenobreak\postnobreak}{\notag}{R_A(x)\defeq\sum_{i=0}^{n} v_i(A)x^i}
\thmend
%
\thmbegin{lem}{\label{lem1}}
$0$,$1$のみを成分に持つ行列$A$に対して, $A$の$i$行$j$列を削除した行列を$B$, $A$の$(i,j)$成分を削除した行列を$C$とすると,
\singledispenv{\prenobreak\postnobreak}{\notag}{R_A(x)=R_B(x)x+R_C(x)\flst}
\thmend
%
\thmbegin{lem}{\label{lem2}}
行列$J_n$を成分がすべて$1$の$n\times n$行列、行列$A$を$0$,$1$のみを成分に持つ$n\times n$行列とすると,
\singledispenv{\prenobreak\postnobreak}{\notag}{\rmfunc{perm}(J_n-A)=\sum_{i=0}^{n}(-1)^i v_i(A)(n-j)!\flst}
\thmend
%
\thmbegin{fact}{\label{fact1}}
\mathenv{\prenobreak\postnobreak}{
  \displine{}{R_{M_1}(x)=x+1}
  \displine{}{R_{M_2}(x)=x^2+3x+1}
  \displine{}{R_{M_3}(x)=x^3+4x^2+4x+1}
  \displine{}{R_{M_4}(x)=x^4+6x^3+11x^2+6x+1}
  \displine{}{R_{M_5}(x)=x^5+9x^4+24x^3+22x^2+8x+1}
  \displine{}{R_{M_6}(x)=x^6+21x^5+70x^4+84x^3+45x^2+11x+1}
  \displine{}{R_{M_7}(x)=x^7+22x^6+91x^5+154x^4+129x^3+56x^2+12x+1}
  \displine{}{R_{M_8}(x)=x^8+23x^7+113x^6+245x^5+283x^4+185x^3+68x^2+13x+1}
  \displine{}{R_{M_9}(x)=x^9+25x^8+158x^7+449x^6+682x^5+597x^4+309x^3+93x^2+15x+1}
  \multiline{}{\showauto{R_{M_{10}}(x)=x^{10}+46x^9+375x^8+1254x^7}\showauto{+2211x^6+2288x^5+1456x^4+575x^3+137x^2+18x+1}}
  \multiline{}{\showauto{R_{M_{11}}(x)=x^{11}+47x^{10}+421x^9+1629x^8+3465x^7}\showauto{+4499x^6+3744x^5+2031x^4+712x^3+155x^2+19x+1}}
  \multiline{}{\showauto{R_{M_{12}}(x)=x^{12}+49x^{11}+514x^{10}+2425x^9+6348x^8+10175x^7}\showauto{+10531x^6+7231x^5+3318x^4+1004x^3+192x^2+21x+1}}
}
\thmend
%
\end{section}
%
\end{document}
